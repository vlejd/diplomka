\documentclass[12pt, oneside]{book}
\usepackage[a4paper,top=2.5cm,bottom=2.5cm,left=3.5cm,right=3.5cm]{geometry}
\usepackage[utf8]{inputenc}
\usepackage[T1]{fontenc}
\usepackage{amsmath}
\usepackage{graphicx}
\usepackage{url}
%\usepackage[slovak]{babel} % vypnite pre prace v anglictine
\usepackage{paralist}
\usepackage{indentfirst}
\usepackage{listings}
\usepackage{wrapfig}
\usepackage{pgfplotstable}
\usepackage{array}
\usepackage[toc,page]{appendix}
\usepackage{hyperref}
\usepackage{cmap} 
\usepackage{amssymb}

\linespread{1.25} % hodnota 1.25 by mala zodpovedat 1.5 riadkovaniu

\newtheorem{definicia}{Definition}[section]

\setcounter{secnumdepth}{3}

\newcommand{\specialcell}[2][c]{%
  \begin{tabular}{{@{}#1@{}}}#2\end{tabular}
}

% -------------------
% --- Definicia zakladnych pojmov
% --- Vyplnte podla vasho zadania
% -------------------
\def\mfrok{2016}
\def\mfnazov{Identifying Differences Between Sequencing Data Sets}
\def\mftyp{Diploma thesis}
\def\mfautor{Vladimír Macko}
\def\mfskolitel{Mgr. Tomáš Vinař, PhD}

%ak mate konzultanta, odkomentujte aj jeho meno na titulnom liste
%\def\mfkonzultant{tit. Meno Priezvisko, tit. }  

\def\mfmiesto{Bratislava, \mfrok}

%aj cislo odboru je povinne a je podla studijneho odboru autora prace
\def\mfodbor{2508 Informatika} 
\def\program{ Informatika }
\def\mfpracovisko{ Katedra informatiky }

\begin{document}     

% -------------------
% --- Obalka ------
% -------------------
\thispagestyle{empty}

\begin{center}
\sc\large
Univerzita Komenského v~Bratislave\\
Fakulta matematiky, fyziky a informatiky

\vfill

{\LARGE\mfnazov}\\
\mftyp
\end{center}

\vfill

{\sc\large 
\noindent \mfrok\\
\mfautor
}

\eject % EOP i
% --- koniec obalky ----

% -------------------
% --- Titulný list
% -------------------

\thispagestyle{empty}
\noindent

\begin{center}
\sc  
\large
Univerzita Komenského v~Bratislave\\
Fakulta matematiky, fyziky a informatiky

\vfill

{\LARGE\mfnazov}\\
\mftyp
\end{center}

\vfill

\noindent
\begin{tabular}{ll}
Study programme: & \program \\
Field of study: & \mfodbor \\
Department: & \mfpracovisko \\
Supervisor: & \mfskolitel \\
% Konzultant: & \mfkonzultant \\
\end{tabular}

\vfill


\noindent \mfmiesto\\
\mfautor

\eject % EOP i


% --- Koniec titulnej strany


% -------------------
% --- Zadanie z AIS
% -------------------
% v tlačenej verzii s podpismi zainteresovaných osôb.
% v elektronickej verzii sa zverejňuje zadanie bez podpisov

\newpage 
\thispagestyle{empty}
%\hspace{-2cm}\includegraphics[width=1.1\textwidth]{images/zadanie} TODO

% --- Koniec zadania

\frontmatter

% -------------------
%   Poďakovanie - nepovinné
% -------------------
\setcounter{page}{3}
\newpage 
~

\vfill
{\bf acknowledgment:}
%TODO

% --- Koniec poďakovania

% -------------------
%   Abstrakt - Slovensky
% -------------------
\newpage 
\section*{Abstrakt}
Zjednodušene môžeme povedať, že sekvenačný beh je množina krátkych podreťazcov vzorkovaných z dlhého neznámeho reťazca. 
Pre dva zadané sekvenačné behy je úlohou identifikovať rozdiely medzi ich zodpovedajúcimi neznámymi reťazcami.

Tento problém bol v minulosti riešený kombinatorickými metódami pre prípady s veľkými sekvenačnými behmi, 
napr. keď množstvo dát v sekvenačnom behu bolo viac ako desať násobok dĺžky neznámeho reťazca. 
Cieľom tejto práce je vytvoriť nové metódy založené na pravdepodobnostnej interpretácii dát. 
Takýto prístup umožné riešiť problém v ďalších prípadoch, napr. pre malé alebo hybridné sekvenačné behy.

\paragraph*{Kľúčové slová:} %TODO
% --- Koniec Abstrakt - Slovensky


% -------------------
% --- Abstrakt - Anglicky 
% -------------------
\newpage 
\section*{Abstract}
In the simplest possible formulation, a sequencing data set is a set of short substrings sampled from an unknown string. Given two sequencing data sets, the goal is to identify differences between the two underlying unknown strings.

This task has been explored by combinatorial methods in scenarios with large sequencing data sets, where amount of data in the set is large (e.g., more than 10x of the length of the underlying string). The goal of this thesis is to develop new methods based on probabilistic interpretation of the data. Such approach may help to solve the problem in other scenarios, including small and hybrid data sets.

\paragraph*{Keywords:} %TODO

% --- Koniec Abstrakt - Anglicky

% -------------------
% --- Predhovor - v informatike sa zvacsa nepouziva
% -------------------
%\newpage 
%\thispagestyle{empty}
%
%\huge{Predhovor}
%\normalsize
%\newline
%Predhovor je všeobecná informácia o práci, obsahuje hlavnú charakteristiku práce 
%a okolnosti jej vzniku. Autor zdôvodní výber témy, stručne informuje o cieľoch 
%a význame práce, spomenie domáci a zahraničný kontext, komu je práca určená, 
%použité metódy, stav poznania; autor stručne charakterizuje svoj prístup a svoje 
%hľadisko. 
% --- Koniec Predhovor


% -------------------
% --- Obsah
% -------------------

\newpage 

\tableofcontents

% ---  Koniec Obsahu

% -------------------
% --- Zoznamy tabuliek, obrázkov - nepovinne
% -------------------

\newpage 

\listoffigures

\listoftables

% ---  Koniec Zoznamov

\mainmatter


\input everything.tex

% -------------------
% --- Bibliografia
% -------------------


\newpage	

\backmatter

\thispagestyle{empty}
\nocite{*}
\clearpage

\bibliographystyle{plain}
\bibliography{literatura} 


%---koniec Referencii

% -------------------
%--- Prilohy---
% -------------------

%Nepovinná časť prílohy obsahuje materiály, ktoré neboli zaradené priamo  do textu. Každá príloha sa začína na novej strane.
%Zoznam príloh je súčasťou obsahu.
%
%\addcontentsline{toc}{chapter}{Appendix A}
\input AppendixA.tex

%\begin{appendices}
%\chapter{Some Appendix}
%\input AppendixA.tex
%\end{appendices}
%
%\addcontentsline{toc}{chapter}{Appendix B}
%\input AppendixB.tex

\end{document}






